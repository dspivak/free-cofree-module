\documentclass[11pt, one side, article]{memoir}


\settrims{0pt}{0pt} % page and stock same size
\settypeblocksize{*}{34.5pc}{*} % {height}{width}{ratio}
\setlrmargins{*}{*}{1} % {spine}{edge}{ratio}
\setulmarginsandblock{.98in}{.98in}{*} % height of typeblock computed
\setheadfoot{\onelineskip}{2\onelineskip} % {headheight}{footskip}
\setheaderspaces{*}{1.5\onelineskip}{*} % {headdrop}{headsep}{ratio}
\checkandfixthelayout


\usepackage{amsthm}
\usepackage{mathtools}

\usepackage[inline]{enumitem}
\usepackage{ifthen}
\usepackage[utf8]{inputenc} %allows non-ascii in bib file
\usepackage{xcolor}

\usepackage[backend=biber, backref=true, maxbibnames = 10, style = alphabetic]{biblatex}
\usepackage[bookmarks=true, colorlinks=true, linkcolor=blue!50!black,
citecolor=orange!50!black, urlcolor=orange!50!black, pdfencoding=unicode]{hyperref}
\usepackage[capitalize]{cleveref}

\usepackage{tikz}

\usepackage{amssymb}
\usepackage{newpxtext}
\usepackage[varg,bigdelims]{newpxmath}
\usepackage{mathrsfs}
\usepackage{dutchcal}
\usepackage{fontawesome}



% cleveref %
  \newcommand{\creflastconjunction}{, and\nobreakspace} % serial comma
  \crefformat{enumi}{\##2#1#3}
  \crefalias{chapter}{section}


% biblatex %
  \addbibresource{Library20211112.bib} 

% hyperref %
  \hypersetup{final}

% enumitem %
  \setlist{nosep}
  \setlistdepth{6}



% tikz %



  \usetikzlibrary{ 
  	cd,
  	math,
  	decorations.markings,
		decorations.pathreplacing,
  	positioning,
  	arrows.meta,
  	shapes,
		shadows,
		shadings,
  	calc,
  	fit,
  	quotes,
  	intersections,
    circuits,
    circuits.ee.IEC
  }
  
  \tikzset{
biml/.tip={Glyph[glyph math command=triangleleft, glyph length=.95ex]},
bimr/.tip={Glyph[glyph math command=triangleright, glyph length=.95ex]},
}

\tikzset{
	tick/.style={postaction={
  	decorate,
    decoration={markings, mark=at position 0.5 with
    	{\draw[-] (0,.4ex) -- (0,-.4ex);}}}
  }
} 
\tikzset{
	slash/.style={postaction={
  	decorate,
    decoration={markings, mark=at position 0.5 with
    	{\draw[-] (.3ex,.3ex) -- (-.3ex,-.3ex);}}}
  }
} 

\newcommand{\bito}[1][]{
	\begin{tikzcd}[ampersand replacement=\&, cramped]\ar[r, biml-bimr, "#1"]\&~\end{tikzcd}  
}
\newcommand{\bifrom}[1][]{
	\begin{tikzcd}[ampersand replacement=\&, cramped]\ar[r, bimr-biml, "{#1}"]\&~\end{tikzcd}  
}
\newcommand{\bifromlong}[2][]{
	\begin{tikzcd}[ampersand replacement=\&, column sep=#2, cramped]\ar[r, bimr-biml, "#1"]\&~\end{tikzcd}  
}

% Adjunctions
\newcommand{\adj}[5][30pt]{%[size] Cat L, Left, Right, Cat R.
\begin{tikzcd}[ampersand replacement=\&, column sep=#1]
  #2\ar[r, shift left=7pt, "#3"]
  \ar[r, phantom, "\scriptstyle\Rightarrow"]\&
  #5\ar[l, shift left=7pt, "#4"]
\end{tikzcd}
}

\newcommand{\adjr}[5][30pt]{%[size] Cat R, Right, Left, Cat L.
\begin{tikzcd}[ampersand replacement=\&, column sep=#1]
  #2\ar[r, shift left=7pt, "#3"]\&
  #5\ar[l, shift left=7pt, "#4"]
  \ar[l, phantom, "\scriptstyle\Leftarrow"]
\end{tikzcd}
}

\newcommand{\xtickar}[1]{\begin{tikzcd}[baseline=-0.5ex,cramped,sep=small,ampersand 
replacement=\&]{}\ar[r,tick, "{#1}"]\&{}\end{tikzcd}}
\newcommand{\xslashar}[1]{\begin{tikzcd}[baseline=-0.5ex,cramped,sep=small,ampersand 
replacement=\&]{}\ar[r,tick, "{#1}"]\&{}\end{tikzcd}}



  
  % amsthm %
\theoremstyle{definition}
\newtheorem{definitionx}{Definition}[chapter]
\newtheorem{examplex}[definitionx]{Example}
\newtheorem{remarkx}[definitionx]{Remark}
\newtheorem{notation}[definitionx]{Notation}


\theoremstyle{plain}

\newtheorem{theorem}[definitionx]{Theorem}
\newtheorem{proposition}[definitionx]{Proposition}
\newtheorem{corollary}[definitionx]{Corollary}
\newtheorem{lemma}[definitionx]{Lemma}
\newtheorem{warning}[definitionx]{Warning}
\newtheorem*{theorem*}{Theorem}
\newtheorem*{proposition*}{Proposition}
\newtheorem*{corollary*}{Corollary}
\newtheorem*{lemma*}{Lemma}
\newtheorem*{warning*}{Warning}
%\theoremstyle{definition}
%\newtheorem{definition}[theorem]{Definition}
%\newtheorem{construction}[theorem]{Construction}

\newenvironment{example}
  {\pushQED{\qed}\renewcommand{\qedsymbol}{$\lozenge$}\examplex}
  {\popQED\endexamplex}
  
 \newenvironment{remark}
  {\pushQED{\qed}\renewcommand{\qedsymbol}{$\lozenge$}\remarkx}
  {\popQED\endremarkx}
  
  \newenvironment{definition}
  {\pushQED{\qed}\renewcommand{\qedsymbol}{$\lozenge$}\definitionx}
  {\popQED\enddefinitionx} 

    
%-------- Single symbols --------%
	
\DeclareSymbolFont{stmry}{U}{stmry}{m}{n}
\DeclareMathSymbol\fatsemi\mathop{stmry}{"23}

\DeclareFontFamily{U}{mathx}{\hyphenchar\font45}
\DeclareFontShape{U}{mathx}{m}{n}{
      <5> <6> <7> <8> <9> <10>
      <10.95> <12> <14.4> <17.28> <20.74> <24.88>
      mathx10
      }{}
\DeclareSymbolFont{mathx}{U}{mathx}{m}{n}
\DeclareFontSubstitution{U}{mathx}{m}{n}
\DeclareMathAccent{\widecheck}{0}{mathx}{"71}


%-------- Renewed commands --------%

\renewcommand{\ss}{\subseteq}

%-------- Other Macros --------%


\DeclarePairedDelimiter{\present}{\langle}{\rangle}
\DeclarePairedDelimiter{\copair}{[}{]}
\DeclarePairedDelimiter{\floor}{\lfloor}{\rfloor}
\DeclarePairedDelimiter{\ceil}{\lceil}{\rceil}
\DeclarePairedDelimiter{\corners}{\ulcorner}{\urcorner}
\DeclarePairedDelimiter{\ihom}{[}{]}

\DeclareMathOperator{\Hom}{Hom}
\DeclareMathOperator{\Mor}{Mor}
\DeclareMathOperator{\dom}{dom}
\DeclareMathOperator{\cod}{cod}
\DeclareMathOperator{\idy}{idy}
\DeclareMathOperator{\comp}{com}
\DeclareMathOperator*{\colim}{colim}
\DeclareMathOperator{\im}{im}
\DeclareMathOperator{\ob}{Ob}
\DeclareMathOperator{\Tr}{Tr}
\DeclareMathOperator{\el}{El}




\newcommand{\const}[1]{\texttt{#1}}%a constant, or named element of a set
\newcommand{\Set}[1]{\mathsf{#1}}%a named set
\newcommand{\ord}[1]{\mathsf{#1}}%an ordinal
\newcommand{\cat}[1]{\mathcal{#1}}%a generic category
\newcommand{\Cat}[1]{\mathbf{#1}}%a named category
\newcommand{\fun}[1]{\mathrm{#1}}%a function
\newcommand{\Fun}[1]{\mathrm{#1}}%a named functor




\newcommand{\id}{\mathrm{id}}
\newcommand{\then}{\mathbin{\fatsemi}}

\newcommand{\cocolon}{:\!}


\newcommand{\iso}{\cong}
\newcommand{\too}{\longrightarrow}
\newcommand{\tto}{\rightrightarrows}
\newcommand{\To}[2][]{\xrightarrow[#1]{#2}}
\renewcommand{\Mapsto}[1]{\xmapsto{#1}}
\newcommand{\Tto}[3][13pt]{\begin{tikzcd}[sep=#1, cramped, ampersand replacement=\&, text height=1ex, text depth=.3ex]\ar[r, shift left=2pt, "#2"]\ar[r, shift right=2pt, "#3"']\&{}\end{tikzcd}}
\newcommand{\Too}[1]{\xrightarrow{\;\;#1\;\;}}
\newcommand{\from}{\leftarrow}
\newcommand{\ffrom}{\leftleftarrows}
\newcommand{\From}[1]{\xleftarrow{#1}}
\newcommand{\Fromm}[1]{\xleftarrow{\;\;#1\;\;}}
\newcommand{\surj}{\twoheadrightarrow}
\newcommand{\inj}{\rightarrowtail}
\newcommand{\wavyto}{\rightsquigarrow}
\newcommand{\lollipop}{\multimap}
\newcommand{\imp}{\Rightarrow}
\renewcommand{\iff}{\Leftrightarrow}
\newcommand{\down}{\mathbin{\downarrow}}
\newcommand{\fromto}{\leftrightarrows}
\newcommand{\tickar}{\xtickar{}}
\newcommand{\slashar}{\xslashar{}}



\newcommand{\inv}{^{-1}}
\newcommand{\op}{^\tn{op}}

\newcommand{\tn}[1]{\textnormal{#1}}
\newcommand{\ol}[1]{\overline{#1}}
\newcommand{\ul}[1]{\underline{#1}}
\newcommand{\wt}[1]{\widetilde{#1}}
\newcommand{\wh}[1]{\widehat{#1}}
\newcommand{\wc}[1]{\widecheck{#1}}
\newcommand{\ubar}[1]{\underaccent{\bar}{#1}}

\newcommand{\dual}[1]{{#1}^\vee}



\newcommand{\bb}{\mathbb{B}}
\newcommand{\cc}{\mathbb{C}}
\newcommand{\nn}{\mathbb{N}}
\newcommand{\pp}{\mathbb{P}}
\newcommand{\qq}{\mathbb{Q}}
\newcommand{\zz}{\mathbb{Z}}
\newcommand{\rr}{\mathbb{R}}


\newcommand{\finset}{\Cat{Fin}}
\newcommand{\smset}{\Cat{Set}}
\newcommand{\smcat}{\Cat{Cat}}
\newcommand{\catsharp}{\Cat{Cat}^{\sharp}}
\newcommand{\ppolyfun}{\mathbb{P}\Cat{olyFun}}
\newcommand{\ccatsharp}{\mathbb{C}\Cat{at}^{\sharp}}
\newcommand{\ccatsharpdisc}{\mathbb{C}\Cat{at}^{\sharp}_{\tn{disc}}}
\newcommand{\ccatsharplin}{\mathbb{C}\Cat{at}^{\sharp}_{\tn{lin}}}
\newcommand{\ccatsharpdisccon}{\mathbb{C}\Cat{at}^{\sharp}_{\tn{disc,con}}}
\newcommand{\sspan}{\mathbb{S}\Cat{pan}}

\newcommand{\List}{\Fun{List}}
\newcommand{\set}{\tn{-}\Cat{Set}}

\newcommand{\plpl}{\mathbin{+\!\!+}}


\newcommand{\yon}{\mathcal{y}}
\newcommand{\poly}{\Cat{Poly}}
\newcommand{\tri}{\mathbin{\triangleleft}}
\newcommand{\tripow}[1]{^{\tri #1}}
\newcommand{\free}{\mathfrak{m}}
\newcommand{\cofree}{\mathfrak{c}}
\newcommand{\Cofree}{\mathfrak{C}}
\newcommand{\tree}{\Set{Tree}}

\newcommand{\rt}{\texttt{root}}
\newcommand{\br}{\tn{\texttt{branch}}}
\newcommand{\hd}{\texttt{head}}
\newcommand{\tl}{\texttt{tail}}

% lenses
\newcommand{\biglens}[2]{
     \begin{bmatrix}{\vphantom{f_f^f}#2} \\ {\vphantom{f_f^f}#1} \end{bmatrix}
}
\newcommand{\littlelens}[2]{
     \begin{bsmallmatrix}{\vphantom{f}#2} \\ {\vphantom{f}#1} \end{bsmallmatrix}
}
\newcommand{\lens}[2]{
  \relax\if@display
     \biglens{#1}{#2}
  \else
     \littlelens{#1}{#2}
  \fi
}



\newcommand{\qand}{\quad\text{and}\quad}
\newcommand{\qqand}{\qquad\text{and}\qquad}


\newcommand{\coto}{\nrightarrow}
\newcommand{\cofun}{{\raisebox{2pt}{\resizebox{2.5pt}{2.5pt}{$\setminus$}}}}
\newcommand{\coh}[1]{^{(#1)}}

\newcommand{\coalg}{\tn{-}\Cat{Coalg}}

\newcommand{\bic}[2]{{}_{#1}\Cat{Comod}_{#2}}

% ---- Changeable document parameters ---- %

\linespread{1.1}
%\allowdisplaybreaks
\setsecnumdepth{section}
\settocdepth{section}
\setlength{\parindent}{15pt}
\setcounter{tocdepth}{1}



%--------------- Document ---------------%
\begin{document}

\title{The free monad monad as a module over\\the cofree comonad comonad in $\poly$}

\author{David I. Spivak}

\date{\vspace{-.2in}}

\maketitle

\begin{abstract}


\smallskip

\noindent
\textbf{Keywords:} 
\end{abstract}


%-------- Chapter --------%
\chapter{Introduction}

See Phil Freeman, \href{https://functorial.com/the-future-is-comonadic/main.pdf}{Declarative UIs are the Future — And the Future is Comonadic!}.

%---- Section ----%
\section{Notation for polynomial functors}

$p(1)$.

%-------- Chapter --------%
\chapter{The cofree comonad comonad $\cofree$}

%---- Section ----%
\section{Behavior trees and initialized coalgebras}

Let $p:\poly$ be a polynomial functor. To understand the cofree comonad $\cofree_p$ on it, one begins with the following definition.

\begin{definition}
Let $p:\poly$ be a polynomial functor. The set $\tree_p$ of \emph{$p$-behavior trees} is defined coinductively as follows. An element $T:\tree_p$ consists of a pair $(I,T.)$ where $I:p(1)$ is a position of $p$ and $T\then -\colon p[I]\to\tree_p$ is a function, i.e.\ for each $i:p[I]$ it returns $T\then i:\tree_p$. 
We refer to $I$ as the \emph{root} of $T$ and denote it $T.\rt$, we write $T.\br\coloneqq p[I]$ and refer to each $i:T.\br$ as a \emph{$T$-branch}, the function $T\then-$ as the \emph{following function}, and the tree $T\then i$ as the \emph{$p$-behavior tree following $i$}.
\end{definition}

\begin{example}
Here is a table of $\tree_p$'s for various polynomials arising from sets $A,B:\smset$.
\[
\begin{array}{l|l|l}
p:\poly&\tree_p&\text{Comment}\\\hline
A&A&\text{$A$-constant trees: $A$-many roots, no branches}\\
A\yon&A^\nn&\text{$A$-Streams: $A$-many roots, one branch each}\\
A\yon+B&A^\nn+B\List(A)&\text{$A$-streams that can terminate in a $B$}\\
A\yon^B&A^{\List(B)}&\text{$B$-branching $A$-trees: print an $A$ and branch by $B$}
\end{array}
\]
\end{example}

\begin{definition}[Cofree category $\Cofree_p$]
Let $p:\poly$ be a polynomial. Consider the graph with vertices $\tree_p$ and for any $T:\tree_p$ a single outgoing edge for each $i:T.\br$, whose target vertex is $T\then i$. Define the \emph{cofree category on $p$}, denoted $\Cofree_p$ to be the free category on that graph.

For any $T:\ob(\Cofree_p)$, we denote the set of outgoing morphisms (paths in the underlying graph) by
\[
\Cofree_p[T]\coloneqq\sum_{T':\tree_p}\Cofree_p(T,T').
\qedhere
\]
\end{definition}

The following is straightforward.
\begin{lemma}
Let $p:\poly$ and $T:\tree_p$ a $p$-behavior tree. Then an element of $\Cofree_p[T]$ can be identified with a list $(i_1,\cdots,i_n)$ where
\[
i_1:T.\br,\quad
i_2:(T\then i_1).\br,\quad...\quad
i_n:(T\then i_1\then\ldots\then i_{n-1}).\br.
\]
The codomain function $\cod\colon\Cofree_p[T]\to\tree_p$ acts on an element $\ell\coloneqq(i_1,\ldots,i_n):\Cofree_p[T]$ by 
\[
\cod\ell\coloneqq T\then i_1\then\cdots\then i_n.
\]
In particular, $\Cofree_p(T,T')\cong\{\ell\colon\Cofree_p[T]\mid\cod(\ell)=T'\}$. 

The identity on $T$ is identified with the length-0 list $()$. Given morphisms $\ell\colon T\to T'$ and $\ell'\colon T'\to T''$ in $\Cofree_p$, their composite, denoted $(\ell\plpl\ell')\colon T\to T''$, is given by concatenation.
\end{lemma}

There is a map $T.\br\to\Cofree_p[T]$ given by sending branches to length-1 lists $i\mapsto (i)$. We will use $\cod$ and $\plpl$ to denote their restriction along this map. In particular, given $\ell\colon T\to T'$ and $i:T'.\br$, we denote their composite by $\ell\plpl i\coloneqq\ell\plpl(i)$.

\begin{definition}
A $p$-coalgebra is a pair $(S,f)$ where $S:\smset$ and $f\colon S\to p(S)$. We refer to elements $s:S$ as \emph{states}. Given a state $s:S$, we can identify $f(s):p(S)$ with a pair
\[
f(s)=(\hd_{f,s},\tl_{f,s}),\quad\text{where}\quad \hd_{f,s}:p(1)\text{ and } \tl_{f,s}\colon p[I]\to S
\]

We refer to a $p$-algebra equipped with a state $s:S$ as an \emph{intialized $p$-coalgebra} and denote it $(S,f,s)$.
 
A \emph{morphism} of $p$-coalgebras $(S,f)\to(S',f')$ consists of a function $g\colon S\to S'$ such that the following diagram commutes:
\[
\begin{tikzcd}
	S\ar[r, "f"]\ar[d, "g"']&p(S)\ar[d, "p(g)"]\\
	S'\ar[r, "f'"']&p(S')
\end{tikzcd}
\]
In particular, the following equations hold:
\[
(\hd_{f,s}=\hd_{f',g(s)}):p(1)
\qqand
(g\circ\tl_{f,s}=\tl_{f',g(s)})\colon p[\hd_{f,s}]\to S'.
\]
A \emph{morphism $(S,f,s)\to(S',f',s')$ of intitialized $p$-coalgebras} is a morphism $g$ of $p$-coalgebras, such that $g(s)=s'$. 

We denote the category of $p$-coalgebras by $p\coalg$ and that of initialized $p$-coalgebras by $p\coalg_*$.
\end{definition}

\begin{proposition}
There is a retraction of categories
\[
\tree_p\To{A} p\coalg_*\To{B}\tree_p,\]
where $\tree_p:\smset$ is considered as a discrete category.
\end{proposition}
\begin{proof}
Given a $p$-tree $T:\tree_p$, consider the set $\Cofree_p[T]:\smset$ of outgoing paths and the element $():\Cofree_p[T]$ given by the $0$-length path. Consider the function $(\cod,\plpl)\colon \Cofree_p[T]\to p(\Cofree_p[T])$, which sends $\ell$ to $(\cod\ell,i\mapsto \ell\plpl(i))$.
Then
\[A(T)\coloneqq(\Cofree_p[T],(\cod,\plpl),())\]
is an initialized $p$-coalgebra.

We now give a functor from $B\colon p\coalg$ to $\tree_p$. On objects, $B$ sends an initialized $p$-coalgebra $(S,f,s)$ to the $p$-behavior tree defined coinductively as
\[
	B(S,f,s)\coloneqq(\hd_{f,s},B(S,f,\tl_{f,s}))
\]
In other words, its root is $\hd_{f,s}:p(1)$ and for any branch $i:p[\hd_{f,s}]$, the $p$-behavior tree following $i$ is $B(S,f,\tl_{f,s}(i))$. Given a morphism $g\colon (S,f,s)\to (S',f',g(s))$, we show $B(S,f,s)=B(S',f',g(s))$ coinductively: the base case is $\hd_{f,s}=\hd_{f',s'}$ and the coinductive step is
\[
B(S,f,\tl_{f,s})=B(S',f',g\circ\tl_{f,s})=B(S',f',\tl_{f',g(s)}).
\]
Thus every morphism in $p\coalg_*$ is sent to an identity. 

The composite of $A$ and $B$ is clearly identity on $\tree_p$ because $\hd_{(\cod,\plpl),()}=\cod()=T.\rt$ and $\tl_{(\cod,\plpl),()}(i)=()\plpl i=T\then i$.
\end{proof}

\begin{proposition}
There is a functor $\cofree_-\colon\poly\to\poly$ such that $\cofree_p$ has the structure of a $\tri$-comonoid for each $p:\poly$,
\[
\cofree_p\to\yon
\qqand
\cofree_p\to\cofree_p\tri\cofree_p.
\qedhere
\]
\end{proposition}
\begin{proof}
Given a polynomial $p$, define polynomials $p^{(i)}$ for $i\in\nn$ by
\[
  p\coh{0}\coloneqq\yon
  \qqand
  p\coh{1+i}\coloneqq\yon\times\left(p\tri p\coh{i}\right)
\]
There is a projection map $\pi\coh{0}\colon p\coh{1}\to p\coh{0}$, and if $\pi\coh{i}\colon p\coh{1+i}\to p\coh{i}$ has been defined, then we can define $\pi\coh{1+i}\coloneqq \yon\times(p\tri\pi\coh{i})$. Now define the polynomial
\begin{equation}\label{eqn.construct_cofree}
\cofree_p\coloneqq\lim\big(\cdots\To{\pi\coh{2}}p\coh{2}\To{\pi\coh{1}}p\coh{1}\To{\pi\coh{0}}p\coh{0}\big)
\end{equation}

This polynomial comes equipped with a counit $\cofree_p\to\yon=p\coh{0}$ given by the projection. We next construct the comultiplication $\cofree_p\to\cofree_p\tri\cofree_p$. Since $\tri$ commutes with connected limits, we have
\[
  \cofree_p\tri\cofree_p=
  \left(\lim_{i_1}p\coh{i_1}\right)\tri\left(\lim_{i_2}p\coh{i_2}\right)=
  \lim_{i_1,i_2}\left(p\coh{i_1}\tri p\coh{i_2}\right)
\]
To obtain the comultiplication $\lim_ip\coh{i}\to\lim_{i_1,i_2}(p\coh{i_1}\tri p\coh{i_2})$, it suffices to produce a natural choice of polynomial map $\varphi_{i_1,i_2}\colon p\coh{i_1+i_2}\to p\coh{i_1}\tri p\coh{i_2}$ for any $i_1,i_2:\nn$. When $i_1=0$ or $i_2=0$, we use the unit identity for $\tri$. By induction, assume given $\varphi_{i_1,1+i_2}$; we construct $\varphi_{1+i_1,1+i_2}$ as follows:
\begin{align}
\nonumber
  p\coh{1+i_1,1+i_2}&=
  \yon\times \left(p\tri p\coh{i_1,1+i_2}\right)\\&\to
\label{eqn.induction}
  \yon\times \left(p\tri p\coh{i_1}\tri p\coh{1+i_2}\right)\\&\to
\label{eqn.special}
  \left(\yon\times p\tri p\coh{i_1}\right)\tri p\coh{1+i_2}\\&=
\nonumber
  p\coh{1+i_1}\tri p\coh{1+i_2}
\end{align}
where \eqref{eqn.induction} is $\varphi_{i_1,1+i_2}$ and it remains to construct \eqref{eqn.special}. Recall that $-\tri q$ preserves products for any $q$, so constructing \eqref{eqn.special} is equivalent to constructing two maps
\[
\yon\times \left(p\tri p\coh{i_1}\tri p\coh{1+i_2}\right)\To{\phi\coh{i_1,i_2}} p\coh{1+i_2}
\qqand
\yon\times \left(p\tri p\coh{i_1}\tri p\coh{1+i_2}\right)\to p\tri p\coh{i_1}\tri p\coh{1+i_2}.
\]
For the latter we use the second projection. The former, $\phi\coh{i_1,i_2}\colon p\coh{1+i_1+1+i_2}\to p\coh{1+i_2}$, is the more interesting one; for it we also use projections $p\coh{i_1}\to p\coh{0}=\yon$ and $\pi\coh{i_2}\colon p\coh{i_2+1}\to p\coh{i_2}$ to obtain:
\[
\yon\times \left(p\tri p\coh{i_1}\tri p\coh{1+i_2}\right)\to
\yon\times \left(p\tri\yon\tri p\coh{i_2}\right)\cong p\coh{1+i_2}
\]
We leave the naturality of this to the reader.

It remains to check that $\epsilon$ and $\delta$ satisfy unitality and coassociativity. The base cases above imply unitality. Proving coassociativity amounts to proving that the following diagram commutes:
\[
\begin{tikzcd}[column sep=50pt]
	p\coh{1+i_1+1+i_2+1+i_3}\ar[r, "\phi\coh{i_1,i_2+1+i_3}"]\ar[d, "\phi\coh{i_1+1+i_2,i_3}"']&
	p\coh{1+i_2+1+i_3}\ar[d, "\phi\coh{i_2,i_3}"]\\
	p\coh{1+i_3}\ar[r,equal]&p\coh{1+i_3}
\end{tikzcd}
\]
This can be shown by induction on $i_3$.
\end{proof}

\begin{lemma}\label{lemma.copointed}
The functor $\cofree_-\colon\poly\to\poly$ is a copointed endofunctor on $\poly$, i.e.\ there is a natural map in $\poly$:
\[\epsilon_p\colon\cofree_p\to p.\]
\end{lemma}
\begin{proof}
By construction \eqref{eqn.construct_cofree} we have projection map $\cofree_p\to p\coh{1}$, and we obtain $\epsilon$ by composing with the projection $p\coh{1}\cong\yon\times p\to p$.
\end{proof}

\begin{theorem}
There is an adjunction
\[
\adj{\catsharp}{U}{\cofree_-}{\poly}
\]
where $U\colon\catsharp\to\poly$ is the forgetful functor $U(c,\epsilon,\delta)=c$.
\end{theorem}
\begin{proof}
We will abuse notation and denote the comonoid $(c,\epsilon,\delta):\catsharp$ simply by its carrier $c$. Our job is to produce a natural isomorphism
\[
\catsharp(c,\cofree_p)\To{\cong}\poly(c,p).
\]
For the forwards direction, we  compose any cofunctor $c\coto\cofree_p$ with the copointing from \cref{lemma.copointed} to obtain a polynomial map $c\to p$. 

For the backwards direction, suppose given a polynomial map $\varphi\colon c\to p$. Since $\cofree_p$ is constructed as a limit \eqref{eqn.construct_cofree}, if we can produce maps $\phi\coh{i}\colon c\to p\coh{i}$ for each $i:\nn$, commuting with the projections $\pi\coh{i}\colon p\coh{1+i}\to p\coh{i}$, then we will have a map $\phi\colon c\to\cofree_p$. Noting that $p\coh{0}=\yon$, we define
\[
\phi\coh{0}\coloneqq\epsilon
\qqand
\phi\coh{1}\coloneqq(\epsilon,\varphi)\colon c\to\yon\times p.
\]
Clearly, we have $\phi\coh{0}=\pi\coh{0}\circ\phi\coh{1}$. Now suppose given $\phi\coh{1+i}\colon c\to p\coh{1+i}$ satisfying $\phi\coh{i}=\pi\coh{i}\circ\phi\coh{1+i}$. Then we define $\phi\coh{i+2}$ as the composite
\[
c\To{(\epsilon,\delta)}\yon\times(c\tri c)\To{\yon\times(\varphi\tri\phi\coh{1+i})}\yon\times\left(p\tri p\coh{1+i}\right)=p\coh{2+i}
\]
and it again satisfies $\phi\coh{1+i}=\pi\coh{1+i}\circ\phi\coh{2+i}$. Thus we have constructed $\phi$.

We need to show that the round-trips on $\catsharp(c,\cofree_p)$ and $\poly(c,p)$ are identities. The latter is straightforward. So suppose given a cofunctor $\psi\colon c\coto\cofree_p$, let $\varphi\coloneqq\epsilon\circ\psi$ and construct $\phi$ as above; it remains to verify that $\psi=^?\phi$.

**
\end{proof}

%-------- Chapter --------%
\chapter{The free monad monad $\free$}

%-------- Chapter --------%
\chapter{The module structure $\free_p\otimes\cofree_q\to\free[p\otimes q]$}


\printbibliography
\end{document}
